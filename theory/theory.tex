\documentclass[12pt]{report}
\usepackage[a4paper]{geometry}
                		% See geometry.pdf to learn the layout options. There are lots.
\geometry{a4paper}
\usepackage{listings}
\usepackage[cm]{fullpage}
\usepackage{layout}
\usepackage{amsthm}
\usepackage{amssymb,amsmath,amsfonts,latexsym,dsfont}

\usepackage{ upgreek }
\usepackage{xcolor}
\usepackage{titlesec}
\usepackage{mathrsfs}
\usepackage{mathtools}
\usepackage[warn]{mathtext}
\usepackage[T1,T2A]{fontenc}
\usepackage{titlesec, blindtext, color}
\definecolor{gray75}{gray}{0.75}
\newcommand{\hsp}{\hspace{20pt}}
\usepackage[utf8]{inputenc}
\usepackage{fancyhdr}
\usepackage[parfill]{parskip}
\usepackage[english,bulgarian,ukrainian,russian]{babel}
\titleformat{\section}[block]{\color{black}\Large\bfseries\filcenter}{}{1em}{}
\titleformat{\chapter}[hang]{\Huge\bfseries}{\thechapter\hsp\textcolor{gray75}{|}\hsp}{0pt}{\Huge\bfseries}
\setcounter{secnumdepth}{0}
\renewcommand{\le}{\leqslant} 
\renewcommand{\ge}{\geqslant }
\DeclareMathOperator{\sign}{sign}
\DeclareMathOperator*{\argmax}{arg\,max}
\DeclareMathOperator*{\argmin}{arg\,min}
\DeclareMathOperator{\Tr}{Tr}
\DeclareMathOperator{\rg}{rg}
\DeclareMathOperator{\diag}{diag}
\DeclareMathOperator{\cov}{cov}
\DeclareMathOperator{\proj}{proj}

           		% ... or a4paper or a5paper or ... 
%\geometry{landscape}                		% Activate for rotated page geometry
%\usepackage[parfill]{parskip}    		% Activate to begin paragraphs with an empty line rather than an indent
\ifx\pdfoutput\undefined
\usepackage{graphicx}
\else
\usepackage[pdftex]{graphicx}
\lstset{language=C++}
\usepackage{pgffor}
\newcounter{SortListTotal}
\newcommand{\sortitem}[2]{\expandafter\def\csname SortListItem#1\endcsname{#2}\stepcounter{SortListTotal}}
\newcommand{\printsortlist}{\foreach\currentlistitem in{1,2,...,\value{SortListTotal}}{\item[\currentlistitem]\csname SortListItem\currentlistitem\endcsname}\setcounter{SortListTotal}{0}}
\newcommand\setItemnumber[1]{\setcounter{enumi}{\numexpr#1-1\relax}}

\title{Метод Монте-Карло для моделирования ферромагнетиков}
\author{Никита Балаганский, Артем Ямалутдинов}
\date{\today}

\usepackage{natbib}
\usepackage{graphicx}
\renewenvironment{proof}{{\bfseries Доказательство:}}{$\square$\\\\}
\newenvironment{solution}{{\bfseries Решение:}}{$\square$\\\\}
\newtheorem{theorem}{Теорема}
\newtheorem{lemma}{Лемма}
\newtheorem{proposition}{Утверждение}
\newtheorem{corollary}{Следствие}
\theoremstyle{definition}
\newtheorem{definition}{Определение}
\newtheorem{notation}{Обозначение}
\newtheorem{example}{Пример}
\newtheorem{problem}{Задача}
\newtheorem{sense}{Смысл}
\newtheorem{remark}{Замечание}
\newcommand{\vect}[1]{\boldsymbol{#1}}

\begin{document}
\maketitle
\fancyhead[C]{field}
\fancyfoot[C]{МФТИ}%
\thispagestyle{fancy}
\newpage
\chapter{1. Введение}
\section{1.1. Ферромагнетизм}
\emph{Ферромагнетиками} называются твердые тела, которые могут обладать \emph{спонтанной намагниченностью}, т. е. намагничены уже в отсутствие магнитного поля.
Типичными представителями ферромагнетиков являются железо, кобальт, никель и многие их сплавы. Ферромагнетизмом обладают многие редкоземельные элементы при пониженной температуре.

Намагниченность ферромагнетиков объясняется тем, что атомы ферромагнетика обладают магнитными моментами и взаимодействуют между собой с силами, зависящими от угла между этими моментами.
Эти силы стремятся установить магнитные моменты соседних атомов параллельно друг другу.

\section{1.2. Модель Изинга}
Модель Изинга --- одна из основных моделей, описывающая поведение магнитных моментов 
в атомах ферромагнетика выше точки Кюри $T_C$. В данной модели каждой вершине кристаллической решетки ставится в соответствие число $+1$ или $-1$, 
отвечаюее за противоположные направления магнитного момента. Это число называется \emph{спином}. Тогда в отсутствие внешнего магнитного поля гамильтониан получившейся системы можно записать в виде
\begin{equation}
    \mathcal{H} = -J\sum_{\langle i, j \rangle}s_is_j,
\end{equation}
где $\langle i, j \rangle$ означает суммирование по соседним элементам,  $J > 0$ --- энергия взаимодействия соседних спинов (считаем ее одинаковой для всех пар),
$s_i, s_j$ --- спины.

При температуре $T_C$ система испытывает \emph{фазовый переход второго рода}.

Покажем, что если кристаллическая решетка одномерная, то фазовый переход в ней невозможен, т.е при любых температурах
упорядоченное состояние энергетически невыгодно.
Для этого запишем \emph{свободную энергию системы}
\begin{equation}
    F = E - TS.
\end{equation}
Здесь  $E$ --- энергия, $S$ --- энтропия системы. Упорядоченное состояние будет устойчиво тогда и только тогда, когда
при переходе в него $\Delta F > 0$.

Изменим спины таким образом, чтобы в системе появился упорядоченный домен. Тогда энергия увеличится на величину $4J$, а изменение энтропии
$\Delta S = k_B \ln N$, где $N$ --- число атомов в решетке. Тогда изменение свободной энергии
\begin{equation}
    \Delta F = 4J - k_B T \ln N < 0 \text{ при }N \to \infty,
\end{equation}
то есть для системы неупорядоченное состояние более выгодно.

Теперь рассмотрим двумерный случай. При введении домена в кристаллическую решетку изменение энергии системы пропорционально периметру домена и может быть записано как
$\Delta E = 4J \cdot \varepsilon N^2, \quad 0 < \varepsilon < 1$. Возможное число доменов можно оценить как  $3^{\varepsilon N^2}$. Тогда изменение энергии
\begin{equation}
    \Delta F = 4J \cdot \varepsilon N^2 - k_B T \ln (N^2 3^{\varepsilon N^2}).
\end{equation}
Отсюда можно получить оценку на точку Кюри $T_C$:
\begin{equation}
    T_C \sim \dfrac{J}{k_B}.
\end{equation}
\chapter{2. Метод Монте-Карло}
\section{2.1. Описание метода}
Для системы из $N$ спинов число возможных состояний равно $2^N$, поэтому
для моделирования систем с достаточно большим $N$ необходимо использовать статистические методы.

В методе Монте-Карло система рассматривается в состоянии термодинамического равновесия при
определенной температуре $T$. В ходе обмена энергией с окружающей средой, энергия будет
флуктуировать около равновесного состояния, а средняя энергия одной частицы пропорциональна $T$.
Тогда для подсчета термодинамических характеристик системы можно будет посчитать ее характеристики для реализованных состояний,
а далее усреднить получившиеся значения.
\section{2.2. Алгоритм Метрополиса}
Описание алгоритма, используемого для моделирования кристаллической решетки ферромагнетика:
\begin{enumerate}
    \item Случайным образом выбирается начальная конфигурация из $N$ спинов $\alpha_0$;
    \item На $k$-ом шаге меняем направление одного случайно выбранного спина и считаем изменение энергии $dE$;
    \item Если $dE < 0$, то новое состояение $\alpha_k$ принимается с вероятностью $1$. 
    Если $dE > 0$, то принимаем новое состояние с вероятностью $P = e^{\frac{-dE}{k_B T}}$;
    \item Повторяем шаги $2-4$, принимая $\alpha_k$ в качестве нового исходного состояния.
\end{enumerate}
\end{document}